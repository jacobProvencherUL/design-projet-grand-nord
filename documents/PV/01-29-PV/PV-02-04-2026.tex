%!TEX encoding = IsoLatin

%
% Exemple d'ordre du jour
% par Pierre Tremblay, Universite Laval
% modifie par Christian Gagne, Universite Laval
% 2011/01/14 - version 1.4
% modifié par Robert Bergevin, Université Laval
% 24/11/2011
% modifié par Jean-Yves Chouinard, Université Laval
% 2016/01/11
% modifié par Jean-Yves Chouinard, Université Laval
% 2017/01/04
%

%--------------------------------------------------------------------------------------
%------------------------------------- preambule --------------------------------------
%--------------------------------------------------------------------------------------
\documentclass[12pt]{ULojpv}

% Chargement des packages supplementaires
\usepackage{lmodern}
\usepackage[T1]{fontenc}
\usepackage[utf8]{inputenc}
\usepackage{enumitem}
\usepackage{pifont}

% Definitions des parametres de l'en-tete
\Cours{GEL--1001 Design I (méthodologie)}             % Nom du cours
\NumeroEquipe{9}                                     % Numero de l'equipe
\NomEquipe{Innovation électrique}                               % Nom de l'equipe
\Objet{Procès-verbal}                                 % Nom du document
\SujetRencontre{Révision et planification}        % Sujet de la rencontre
\DateRencontre{2026/02/05}                            % Date de la rencontre
\LocalRencontre{Microsoft Teams}                            % Local de la rencontre
\HeureRencontre{08h30-10h30}                          % Heure de la rencontre

%--------------------------------------------------------------------------------------
%--------------------------------- corps du document ----------------------------------
%--------------------------------------------------------------------------------------
\begin{document}
\entete
\begin{enumerate}
      \item \textbf{Ouverture de la réunion}

            Heure : 8h30
      \item \textbf{Nomination ou confirmation du président et du secrétaire}
            \\ % saut de ligne
            \begin{tabular}{@{}ll}
                  Président : Loïc Constantineau & Secrétaire : Rosemary St-Pierre
            \end{tabular}
      \item \textbf{Lecture et adoption de l'ordre du jour}
            \\ % saut de ligne
            \begin{tabular}{@{}ll}
                  L'ordre du jour est proposé et adopté à l'unanimité.
            \end{tabular}
      \item \textbf{Lecture et adoption du procès-verbal de la réunion du 29 janvier 2026}
            \\ % saut de ligne
            \begin{tabular}{@{}ll}
                  Aucun procès-verbal n'était dû pour cette rencontre.
            \end{tabular}
      \item \textbf{Affaires découlant du procès-verbal}
      \item \textbf{Points à traiter}
            \begin{enumerate}
                  \item Révision du rapport de projet - version 0

                        Chaque membre de l'équipe a révisé l'introduction et la description écrites par Antoine avec le support de Marika. Quelques commentaires ont été donnés sur les unités de mesure. Jacob a mentionné que les unités de mesure doivent être séparées de la valeur (ex: 2m vs 2 m). Le contenu du rapport de projet - version 0 a été approuvé par l'entièreté de l'équipe.

                  \item Discuter et déterminer les prochaines tâches à réaliser

                        \begin{enumerate}
                              \item Remue-méninges et recherche des composantes pour la boîte

                                    Dans un premier temps, l'équipe a effectué l'attribution de nouveaux rôles sur le contrat d'équipe. Nous nous sommes assurés que chacun se voit assigner un rôle différent que celui des deux premières semaines.

                                    Dans un deuxième temps, l'équipe a fait une mise à jour de l'ensemble du projet dans l'objectif de comprendre quelles seront les prochaines étapes et les prochaines tâches à réaliser pour faire avancer le projet. À la lueur d'une lecture de la description pour la remise du rapport de projet - version 1, il a été possible de répartir les tâches à chaque membre de l'équipe, tout en s'assurant d'avoir une charge de travail équilibrée. Avec Miro, un logiciel d'aide pour l'idéation et le remue-méninge, on a inscrit visuellement les tâches de chacun.

                                    Aucune discussion n'a été abordée au sujet de la recherche des composantes pour la boîte.

                              \item Mettre en place les tâches sur YouTrack et le diagramme de Gantt

                                    L'attribution des tâches a été inscrite dans YouTrack.
                        \end{enumerate}

                  \item Échanger sur nos points forts et nos points faibles afin d'organiser la répartition des tâches de manière plus judicieuse.

                        Jacob a initié une discussion autour de nos points forts et nos points faibles. L'idée était de trouver un rôle qui s'aligne avec les forces de chacun. D'abord, on a trouvé deux personnes avec un certain intérêt à l'égard du français et la grammaire, personnes vers qui on pourra se référer pour des questions sur une structure de phrases, de la grammaire, etc. Ensuite, on a trouvé une personne intéressée à être une référence en LaTeX, Jacob.

                        La rencontre s'est terminée sur cette discussion. Chacun ayant au moins une tâche pour la semaine, avec des dates dues.
            \end{enumerate}

      \item \textbf{Divers}

            Antoine nous a expliqué le fonctionnement lorsqu'on cite nos sources avec Latex. Il a aussi remis à l'équipe un document LaTeX tiré du cégep en guise de documents de référence supplémentaires LaTeX.

      \item \textbf{Répartition des tâches}

            \begin{description}[
                        leftmargin=4.5cm,
                        labelwidth=4.3cm,
                        labelsep=0.2cm,
                        align=left,
                        font=\normalfont
                  ]
                  \item[Hiba:]
                        Produire l'ordre du jour pour la réunion du 5 février et l'envoyer à l'équipe pour une révision finale.
                  \item[Antoine:]
                        Écrire une première version pour la section \textbf{Besoins} dans le rapport de projet pour dimanche 1 février.
                  \item[Marika et Rosemary:]
                        Écrire une première version pour la section \textbf{Objectifs} dans le rapport de projet pour jeudi 5 février.
                  \item[Prince et Loïc:]
                        À partir des objectifs, clarifier une procédure pour établir le cahier des charges.
                  \item[Rogeo:] Mettre à jour le diagramme de Gantt dans YouTrack et s'assurer de la cohérence entre les tâches de la semaine et leur date d'échéance.
                  \item[Jacob:] Produire le procès-verbal avant mercredi 4 février.
            \end{description}

      \item \textbf{Évaluation de la réunion}

            La réunion s'est bien déroulée et a permis de donner à l'équipe une direction et une vision plus claire du travail à réaliser pour les trois prochaines semaines.

      \item \textbf{Date, heure, lieu et objectif de la prochaine réunion}

            \begin{tabular}{@{}lll}
                  Date: 2026/02/05
                   & Heure: 08h30
                   & Lieu: Microsoft Teams
            \end{tabular}

            La prochaine réunion a pour objectif de consolider les besoins et objectifs afin de partir sur de bonnes bases pour l'établissement du cahier des charges, qui s'étendra sur les deux semaines suivantes.

      \item \textbf{Fermeture de la réunion}

            Heure : 10h21

      \item \textbf{Étaient présents}

            \begin{dinglist}{"33}
                  \item Antoine Demers
                  \item Hiba Jettane
                  \item Jacob Provencher
                  \item Loïc Constantineau
                  \item Marika Thibault
                  \item Prince Akissoe
                  \item Rogeo Dounla
                  \item Rosemary St-Pierre
            \end{dinglist}

\end{enumerate}
\end{document}

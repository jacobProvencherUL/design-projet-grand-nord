            
            
            \begin{enumerate}

                  \item Rappel des exigences du rapport de projet – version 1

                        L’équipe a procédé à une révision de la grille de correction du rapport de projet (version 0), afin de bien comprendre les attentes pour la version 1.

                  \item Définition des besoins du client

                        Les commentaires reçus ont été corrigés et la section Besoins a été révisée collectivement.
                        L’équipe s’est questionnée sur le format de présentation (liste versus texte). Le format sous forme de liste a été approuvé, avec la possibilité de le modifier ultérieurement si nécessaire, puisqu’il s’agit d’un choix facilement réversible.

                  \item Définition et hiérarchisation des objectifs

                        Une révision complète de la liste des objectifs a été effectuée.
                        L’équipe s’est assurée que chaque objectif soit clairement formulé, qu’il corresponde bien à un objectif (et non à un besoin ou à un concept), et qu’il utilise des verbes d’action appropriés.
                        Une discussion a eu lieu concernant la distinction entre un besoin et un concept.
                        Prince a soulevé une question concernant la section 4.5, plus précisément la simplification de l’entretien du capteur. Antoine a alors questionné la pertinence de conserver cet aspect.
                        Le tableau des objectifs a été révisé et devra être mis à jour afin de refléter les décisions prises.

                  \item Élaboration du cahier des charges

                        L’équipe a discuté de l’élaboration du cahier des charges, tout en soulignant qu’il manque encore certaines informations pour avancer efficacement.
                        Rosemary a posé une question concernant le nombre minimal de critères à inclure (environ 10 à 12). Il a été convenu que ce nombre servait principalement de référence et que le jugement de l’équipe devait prévaloir.

                  \item Organisation du travail d’équipe

                        L’équipe a discuté des stratégies pour mieux synchroniser les modifications apportées aux besoins, aux objectifs et au cahier des charges.
                        Une rencontre éventuelle dimanche entre Antoine, Prince et Loïc a été envisagée, puis annulée en raison de l’indisponibilité d’Antoine. Une rencontre entre Prince et Loïc reste possible, suivie d’un débriefing à Antoine le mardi.

            \end{enumerate}



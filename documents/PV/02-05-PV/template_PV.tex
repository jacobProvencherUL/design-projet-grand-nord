
\entete
\begin{enumerate}
      \item \textbf{Ouverture de la réunion}

            Heure : 08h30

      \item \textbf{Nomination ou confirmation du président et du secrétaire}

            \begin{tabular}{@{}ll}
                  Président : Antoine Demers & Secrétaire : Marika Thibault
            \end{tabular}

      \item \textbf{Lecture et adoption de l'ordre du jour}

            \begin{tabular}{@{}ll}
                  L'ordre du jour est proposé et adopté à l'unanimité.
            \end{tabular}

      \item \textbf{Lecture et adoption du procès-verbal de la réunion du 29 janvier 2026}

            \begin{tabular}{@{}ll}
                  Le procès-verbal est proposé et adopté à l'unanimité.
            \end{tabular}

      \item \textbf{Affaires découlant du procès-verbal}

      \item \textbf{Points à traiter}

            \begin{enumerate}

                  \item Rappel des exigences du rapport de projet – version 1

                        \begin{itemize}
                              \item On a révisé la grille de correction du rapport version 0
                        \end{itemize}

                  \item Définition des besoins du client

                        \begin{itemize}
                              \item On a ensuite corriger les commentaires données
                              \item Ensuite, on a révisé la section Besoins
                              \item On s'est demandé si le format de la section Besoins était okay sous format liste
                              \item On a approuvé (on pourra changer par après s'il le faut : c'est un choix facilement réversible)
                        \end{itemize}

                  \item Définition et hiérarchisation des objectifs

                        \begin{itemize}
                              \item Ensuite, on a fait une révision de la liste des objectifs
                              \item On s'est assuré que chaque objectif est vraiment un "objectif", c'est-à-dire qu'il est claire et qu'on utilise des verbes d'actions.
                              \item Ensuite, on a discuté sur la différence entre un besoin et un concept.
                              \item Prince a posé une question sur la section 4.5 (plus spécifiquement la section 4.5.3) : il avait une interrogation par rapport à la simplification de l'entretien du capteur.
                              \item Antoine demande si on devrait enlever l'entretien du capteur
                              \item Révision du tableau des objectifs
                              \item Il faudra updater le tableau des objectifs
                        \end{itemize}

                  \item Élaboration du cahier des charges

                        \begin{itemize}
                              \item On espère qu'on aura plus d'informations sur le cahier de charges
                              \item Rosemary demande si dans le cahier de charges, il faut au moins 10 à 12 critères.
                              \item On attend encore d'autres informations pour mieux avancer le cahier de charges.
                        \end{itemize}

                  \item Organisation du travail d’équipe

                        \begin{itemize}
                              \item On essaye de se synchroniser pour bien gérer la modification des besoins, ensuite les objectifs et ensuite le cahier de charges.
                              \item Possible call dimanche entre Antoine, Prince, Loïc?
                              \item Cancel : Antoine sera à MTL
                              \item Possible call entre Prince et Loïc, dimanche
                              \item Mardi : debrief à Antoine par Prince et Loïc
                        \end{itemize}

            \end{enumerate}

      \item \textbf{Divers}

            \begin{enumerate}

                  \item Loïc propose qu'à la fin de la réunion de faire un tour de table pour s'assurer que chacun sait c'est quoi sa ou ses tâches de la semaine.

                  \item Question de Marika : complication lorsqu'on envoie les docs de gestion --> travailler sur Overleaf pour production des ODJ et P-V.

                  \item Ajustement : travailler sur Overleaf pour produire les ODJ et les P-V (car si coéquipiers veulent apporter modifications, ils pourront VS si on envoie un PDF --> impossible de modifier (il faut donner des commentaires instead)

                  \item Rosemary : comment mettre nom de ma figure

                  \item Antoine répond : (essaye de fort de s'en rappeler...) il explique la fonction label

                  \item Conclusion : on demandera à ChatGPT (don't write anywhere else...)

                  \item Brainstorm sur le stockage des données :
                        \begin{itemize}
                              \item Disque dur local
                              \item Cloud
                        \end{itemize}

                  \item Résumé sur la rédaction
                        \begin{itemize}[leftmargin=*]
                              \item \textbf{Mainly :} Loïc et Prince
                              \item \textbf{Helper :} Marika et Antoine
                        \end{itemize}

                  \item Quand Robert est venu dans le call Teams
                        \begin{enumerate}[leftmargin=*]
                              \item \textbf{Question} Questions sur le format des objectifs en format liste? C'est good?

                                    \textbf{Réponse} Semble bien puisqu'on a décrit chaque besoin.

                                    Voir séminaire 8 pour plus d'info

                                    Simon corrigera la partie besoins et objectifs

                                    Parler à Simon pour confirmer notre façon de faire pour présenter les besoins et objectifs.

                                    Maybe : un petit texte pour guider le lecteur. Essayer de simplifier la vie de la personne qui lira notre rapport. Pourquoi X. Pourquoi Y.

                                    Faire un petit paragraphe au début de la section 3.2.

                                    Bref : guider le lecteur!

                                    Question - Antoine : Est-ce qu'on a une liste de qui corrige quoi?

                                    Luc a donné une liste de qui corrige quoi

                              \item \textbf{Question} Vérification sur le nombre de critères pour le cahier de charges. On a vu qu'il faut entre 10 à 12 critères d'évaluations. Est-ce exacte?

                                    \textbf{Réponse} selon notre jugement...

                                    Plus on a de critères, plus il y a une grosse charge de travail. Bref, en avoir assez de sorte à bien cerner le problème. Se fier à notre jugement. 10-12 est surtout un chiffre référence, par expérience...
                        \end{enumerate}

            \end{enumerate}

      \item \textbf{Objectifs de la semaine}

            \begin{itemize}[leftmargin=*]
                  \item Avoir une ébauche du \textbf{cahier de charges} (pour jeudi prochain)
                  \item Modifications des \textbf{besoins} (avant dimanche)
                  \item À partir de mardi, Antoine pourra commencer à rédiger le cahier de charges (Antoine + Prince)
                  \item Impératif : avoir l'ODJ avant mardi (pour révision pour la remise du lendemain)
                  \item Idéalement : avoir procès-verbal avant dimanche (selon Jacob)
                  \item Bien setup les templates pour l'ODJ et le P-V sur Overleaf pour meilleur workflow.
            \end{itemize}

      \item \textbf{Répartition des tâches}

            \begin{description}[
                        leftmargin=4.5cm,
                        labelwidth=4.3cm,
                        labelsep=0.2cm,
                        align=left,
                        font=\normalfont
                  ]

                  \item[Antoine:]
                        \begin{itemize}
                              \item Supervision du projet
                              \item Supervision de l'ODJ
                              \item Un peu de rédaction
                        \end{itemize}
                  \item[Hiba:]
                        Rédaction des ordres du jour (jusqu'à la remise du rapport v1)
                  \item[Jacob:]
                        Rédaction des procès-verbaux (jusqu'à la remise du rapport v1)
                  \item[Marika:]
                        \begin{itemize}
                              \item Remise du livrable
                              \item Écrire du petit texte pour section 3.2: Analyse des objectifs
                        \end{itemize}
                  \item[Loïc et Prince:]
                        Production de la partie rédaction
                  \item[Rogeo:]
                        \begin{itemize}
                              \item S'assurer de garder à jour le diagramme de tâches (ajout de tâches, mise à jour des tâches)
                              \item Ajouter toutes les deadlines jusqu'à la fin de la session (e.g. toutes les dates de remises)
                        \end{itemize}
                  \item[Rosemary:]
                        \begin{itemize}
                              \item Vérification et supervision du rapport
                              \item Aide sur YouTrack pour Rogeo
                        \end{itemize}

            \end{description}

      \item \textbf{Évaluation de la réunion}

            La réunion s'est bien déroulée et a permis de consolider la section des besoins et des objectifs afin de partir sur de bonnes bases pour la rédaction du cahier des charges sur les deux semaines à venir.

      \item \textbf{Date, heure, lieu et objectif de la prochaine réunion}

            \begin{tabular}{@{}lll}
                  Date: 2026/12/02
                   & Heure: 08h30
                   & Lieu: Microsoft Teams
            \end{tabular}

            La prochaine réunion a pour objectif de réviser les modifications apportés aux besoins, aux objectifs et donc aussi sur le cahier des charges.

      \item \textbf{Fermeture de la réunion}

            Heure : 10h12

      \item \textbf{Étaient présents}

            \begin{dinglist}{"33}
                  \item Antoine Demers
                  \item Jacob Provencher
                  \item Loïc Constantineau
                  \item Marika Thibault
                  \item Prince Akissoe
                  \item Rogeo Dounla
                  \item Rosemary St-Pierre
            \end{dinglist}

\end{enumerate}

\begin{enumerate}
    \item Tour de table de fin de réunion
    
    Loïc propose qu’un tour de table soit effectué à la fin de chaque réunion afin de s’assurer que chaque membre comprend clairement ses tâches pour la semaine à venir.

    \item Outils de rédaction des documents de gestion
    
    Marika soulève des difficultés liées à l’envoi des documents de gestion en format PDF.
    Il est convenu que l’équipe travaillera désormais sur Overleaf pour la production des ordres du jour et des procès-verbaux, afin de faciliter la collaboration et les modifications en temps réel.

    \item Mise en forme des figures
    
    Rosemary pose une question concernant l’ajout de noms aux figures dans les documents.
    Antoine explique l’utilisation de la commande \texttt{\textbackslash label}.
    \item Stockage des données
    
    Un échange informel a eu lieu concernant les options de stockage des données, notamment :
    \begin{itemize}
        \item disque dur local ;
        \item stockage infonuagique (cloud).
    \end{itemize}

    \item Organisation de la rédaction
    
    La rédaction principale sera assurée par Loïc et Prince, avec l’aide de Marika et Antoine.

    \item Intervention de Robert
    
    Robert est intervenu lors de la réunion afin de clarifier certains points :
    
    \begin{enumerate}
        \item Format des objectifs
        
        Le format en liste est jugé adéquat puisqu’il permet de bien décrire chaque besoin.
        Il est suggéré d’ajouter un court paragraphe introductif pour guider le lecteur et faciliter la compréhension du rapport.
        Une validation sera effectuée auprès de Simon.

        \item Nombre de critères du cahier des charges
        
        Le nombre de 10 à 12 critères est considéré comme une valeur de référence.
        L’équipe est invitée à se fier à son jugement afin d’inclure suffisamment de critères pour bien cerner le problème, sans alourdir inutilement la charge de travail.
    \end{enumerate}

\end{enumerate}

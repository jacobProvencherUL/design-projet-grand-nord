On a commencé la réunion en révisant collectivement les commentaires données sur l'ordre du jour et le procès-verbale de la remise passée. On s'est assuré d'avoir corriger le tir pour les prochaines remises. Les principales intéressés Jacob et Hiba ainsi que Rosemary ont confirmé avoir lu et ajusté les documents de gestions et le diagramme de Gantt en conséquence.

Ensuite, Marika a initié un tour de table pour le faire point sur ce qui a été fait au courant de la semaine. D'abord, Rosemary avec l'aide de Rogeo avait comme mandat de garder à jour le diagramme de Gantt ainsi que d'y ajouter des tickets pour les dates importantes jusqu'à la fin de la session. Chaque membre de l'équipe en a pris connaissance et a approuvé le travail fait.

Ensuite, on a entamé une discussion sur les besoins et les objectifs. Marika a demandé à l'équipe si c'était incohérent d'avoir un nombre différent pour les besoins et pour les objectifs. L'équipe répond que c'est correcte. Ensuite, elle révise un à un les objectifs, suggérant entre autre le fusionnement des deux premiers sous-objectifs, car ils ont la même signification. L'équipe approuve ce changement.

À la lueur de cette révision, Marika soulève la possibilité d'ajouter un objectif sur les coûts. Rappelant ce qui a été dit au dernier séminaire, avoir un objectif concernant les coûts est primordial. Alors, d'abord, Marika demande quels genre de coûts dont il est question ici. Rosemary propose d'abord les coûts des pièces. Ensuite, Antoine renchérit en mentionnant les coûts d'installation, de fabrication, d'opérations et d'entretien. L'équipe approuve l'ajout d'un tel objectif. Antoine mentionne qu'il sera important de bien décrire les coûts dont il sera question ici, pour s'assurer qu'il n'y ait aucune ambiguïté.

Ensuite Marika finit le tour de table en entamant une discussion sur l'avancement dans la section du cahier des charges. Antoine et Prince, qui en était responsable pour cette semaine, rapportent ce qui a été fait à l'équipe : une introduction a été écrite pour chaque section principale et une simple phrase pour décrire l'intention a été écrite dans chaque sous-section.

Finalement, Marika propose de répartir la rédaction du cahier des charges : puisque c'est une grosse section, et qu'il reste seulement qu'une semaine avant la remise, il est de mis de se séparer la tâche. Ainsi, à tour de rôle, chaque membre est assigné une section à rédiger. La répartition des tâches est inscrite dans la section prévue à cette effet.

Divers

Rosemary mentionne que les contraintes min-max sont juste les besoins, et non sur les objectifs. Elle mentionne également que les objectifs doivent commencer par des verbes d'actions et que les critères d'évaluation ne commencent pas par des verbes d'action. Elle envoie une capture d'écran pour nous rappeler ce qui a été présenté en cours. Finalement, elle mentionne aussi que dans le tableau du cahier des charges, il faut ajouter les unités.

Jacob, au pris à prendre les notes de la réunion, demande de faire un résumé du travail qui est à faire dans la section cahier des charges. Antoine lui répond en utilisant le rapport de projet de référence pour accompagner ses explications. Aussi, Loïc nous a expliqué sa procédure pour établir les relations dans le tableau du cahier des charges. Il a utilisé le logiciel Desmos pour supporter son explication.

Pendant les dernières minutes de la rencontre, Antoine a mis à jour le tableau du cahier des charges afin d'avoir une image complète de l'état actuel du cahier des charges, donnant ainsi une image plus claire de ce qui est à compléter pour la semaine prochaine.

Répartition des tâches

Antoine :
- réviser la section Analyse des objectifs avant dimanche après-midi
- rédiger la section 4.1 du cahier des charges avant dimanche après-midi
Prince :
- rédiger la section 4.2 du cahier des charges avant lundi soir
Loïc :
- rédiger la section 4.3 du cahier des charges avant lundi soir
Hiba :
- rédiger la section 4.4 du cahier des charges avant lundi soir
- rédiger l'ordre du jour avant mardi midi
Marika :
- fusionner les deux premiers sous-objectifs du premier objectif avant dimanche
- rédiger la description pour le nouvel objectif sur les coûts avant dimanche
- rédiger la section 4.5 du cahier des charges avant lundi soir
Rosemary :
- Ajuster le schéma des objectifs en conséquence selon les modifications apportées par Marika sur la liste d'objectifs avant lundi soir
- Produire le tableau de la maison de la qualité avant mardi soir
Rogeo :
- Clarifier le travail à faire pour le rapport de projet - version 2 en écrivant un rapport avant la prochaine réunion, soit le jeudi 19 février.
Jacob :
- rédiger le procès-verbal avant mardi midi





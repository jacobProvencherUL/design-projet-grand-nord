\begin{enumerate}
    \item Révision des modifications faites aux besoins et aux objectifs

          Ensuite, Marika a initié un tour de table pour le faire point sur ce qui a été fait au courant de la semaine. D'abord, Rosemary avec l'aide de Rogeo avait comme mandat de garder à jour le diagramme de Gantt ainsi que d'y ajouter des tickets pour les dates importantes jusqu'à la fin de la session. Chaque membre de l'équipe en a pris connaissance et a approuvé le travail fait.

    \item Révision des modifications faites sur le diagramme de Gantt

          Ensuite, on a entamé une discussion sur les besoins et les objectifs. Marika a demandé à l'équipe si c'était incohérent d'avoir un nombre différent pour les besoins et pour les objectifs. L'équipe répond que c'est correcte. Ensuite, elle révise un à un les objectifs, suggérant entre autre le fusionnement des deux premiers sous-objectifs, car ils ont la même signification. L'équipe approuve ce changement.

          À la lueur de cette révision, Marika soulève la possibilité d'ajouter un objectif sur les coûts. Rappelant ce qui a été dit au dernier séminaire, avoir un objectif concernant les coûts est primordial. Alors, d'abord, Marika demande quels genre de coûts dont il est question ici. Rosemary propose d'abord les coûts des pièces. Ensuite, Antoine renchérit en mentionnant les coûts d'installation, de fabrication, d'opérations et d'entretien. L'équipe approuve l'ajout d'un tel objectif. Antoine mentionne qu'il sera important de bien décrire les coûts dont il sera question ici, pour s'assurer qu'il n'y ait aucune ambiguïté.

    \item Validation de la qualité et de la cohérence des critères d’évaluation

          Ensuite Marika finit le tour de table en entamant une discussion sur l'avancement dans la section du cahier des charges. Antoine et Prince, qui en était responsable pour cette semaine, rapportent ce qui a été fait à l'équipe : une introduction a été écrite pour chaque section principale et une simple phrase pour décrire l'intention a été écrite dans chaque sous-section.

\end{enumerate}

